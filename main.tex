\documentclass{butex}
\policeprincipale{mathpazo}
\policesecondaire{mathpazo}



%-------------------- Informations sur le rapport ----------------------
% Remplir les informations qui seront utilisées sur la page de garde et dans les en-têtes/pieds de page
\titre{Développement}
\UE{Typographie}
\sujet{Projet de typographie avec \LaTeX}
\enseignant{Dr. LaTeX}
\eleves{Letudiant Joel \\ Letudiante Joelle}

%-------------------- Début du contenu du document ----------------------
\begin{document}

% Création de la page de garde
\fairepagedegarde

% Activation des en-têtes et pieds de page
\fairemarges

% Création de la table des matières

\tabledematieres

%======================================================================================
\section{Introduction et concepts de base}
%======================================================================================

Ce document a pour objectif de démontrer les multiples fonctionnalités offertes par la classe \LaTeX{} personnalisée \texttt{rapport.cls}. Chaque section explorera un ensemble de commandes et d'environnements spécifiques pour illustrer leur utilisation et leur rendu visuel.
"The following sections will detail various class commands like \texttt{\textbackslash section} for creating structured headings."
\info{
	Ce document est auto-descriptif. Le code \LaTeX{} qui le génère est un exemple direct de l'utilisation de la classe. N'hésitez pas à consulter le fichier \texttt{.tex} pour voir comment chaque élément est implémenté.
}

Nous allons commencer par les éléments de base tels que les listes et les différents niveaux de titres, avant de passer à des sujets plus complexes. L'utilisation correcte des guillemets typographiques français se fait avec la commande \verb|\enquote{...}|, comme ceci : \enquote{Ceci est un exemple}.

\subsection{Listes à puces et numérotées}

Les listes sont stylisées pour une meilleure lisibilité.

\tsecnonum{Exemple de liste à puces (itemize)}
\begin{itemize}
	\item Premier élément de la liste.
	\item Deuxième élément, qui peut s'étendre sur plusieurs lignes si nécessaire pour démontrer l'indentation.
	\item Troisième élément avec une sous-liste :
	\begin{itemize}
		\item Sous-élément A.
		\item Sous-élément B.
	\end{itemize}
\end{itemize}

\tsecnonum{Exemple de liste numérotée (enumerate)}
\begin{enumerate}
	\item Le premier pas est toujours le plus important.
	\item Le deuxième suit logiquement.
	\item Et ainsi de suite, avec une numérotation claire et en gras.
\end{enumerate}

\subsection{Structure des Titres}
La classe définit un style particulier pour les sections, sous-sections et sous-sous-sections, comme vous pouvez le constater tout au long de ce document. Voici un exemple de la hiérarchie.

\subsubsection{Ceci est un sous-sous-titre}
Le style est plus discret pour indiquer un niveau de détail inférieur.


%======================================================================================
\section{Mathématiques et Environnements Scientifiques}
%======================================================================================
La classe intègre des outils puissants pour la rédaction de contenu scientifique, notamment les mathématiques et les théorèmes.

\subsection{Équations et Formules}

Les équations sont numérotées par section. Voici les équations de Maxwell, un exemple classique utilisant l'environnement \texttt{align}.

\begin{align}
	\nabla \cdot \mathbf{E} &= \frac{\rho}{\varepsilon_0} \label{eq:maxwell1} \\
	\nabla \cdot \mathbf{B} &= 0 \label{eq:maxwell2} \\
	\nabla \times \mathbf{E} &= -\frac{\partial \mathbf{B}}{\partial t} \label{eq:maxwell3} \\
	\nabla \times \mathbf{B} &= \mu_0 \left( \mathbf{J} + \varepsilon_0 \frac{\partial \mathbf{E}}{\partial t} \right) \label{eq:maxwell4}
\end{align}

\info{
	L'équation de Gauss (\ref{eq:maxwell1}) est fondamentale en électromagnétisme.
}

\subsection{Théorèmes, Définitions et Remarques}
Des environnements prédéfinis permettent de structurer le discours scientifique.

\begin{definition}[Groupe]
	Un groupe est un ensemble non vide $G$ muni d'une loi de composition interne $\ast$ qui est associative, admet un élément neutre et pour laquelle chaque élément admet un symétrique.
\end{definition}

\begin{theorem}[Théorème de Lagrange]
	Si $H$ est un sous-groupe d'un groupe fini $G$, alors l'ordre de $H$ divise l'ordre de $G$.
\end{theorem}

\begin{exemple}
	L'ensemble des entiers relatifs $\mathbb{Z}$ muni de l'addition est un groupe.
\end{exemple}

\begin{remarque}
	Toutes ces boîtes partagent un style cohérent pour une lecture agréable.
\end{remarque}

\subsection{Unités Scientifiques}
Le package \texttt{siunitx} est configuré pour le français.

\tsec{Utilisation de \texttt{siunitx}}
La constante de Planck \nomenclature{$h$}{Constante de Planck} est d'environ \num{6.626e-34}.
La vitesse de la lumière\nomenclature{$c$}{Vitesse de la lumière dans le vide} dans le vide est $c = \SI{299792458}{\meter\per\second}$.

%======================================================================================
\section{Éléments Visuels : Figures et Tableaux}
%======================================================================================

\subsection{Insertion de Figures}
La commande personnalisée \verb|\insererfigure| simplifie l'ajout d'images encadrées.

% Utilisation de la commande personnalisée
\insererfigure{logos/logo.png}{4cm}{Ceci est le logo principal, inséré avec notre commande personnalisée \texttt{\textbackslash insererfigure}.}{logo_principal}

Pour des besoins plus complexes, comme les sous-figures, les environnements standards fonctionnent toujours.

\begin{figure}[H]
	\centering
	\begin{subfigure}{0.45\textwidth}
		\centering
		\includegraphics[width=0.8\linewidth]{logos/logo.png}
		\caption{Première sous-figure.}
		\label{fig:sub1}
	\end{subfigure}
	\hfill % Espace entre les deux figures
	\begin{subfigure}{0.45\textwidth}
		\centering
		\includegraphics[width=0.8\linewidth]{logos/logo_ECL.jpg}
		\caption{Deuxième sous-figure.}
		\label{fig:sub2}
	\end{subfigure}
	\caption{Exemple de figure avec deux sous-figures utilisant le package \texttt{subcaption}.}
	\label{fig:sousfigures}
\end{figure}

\subsection{Création de Tableaux}
Les tableaux sont stylisés avec \texttt{booktabs} pour un rendu professionnel.

\begin{table}[H]
	\centering
	\caption{Comparaison des caractéristiques de différents langages.}
	\label{tab:langages}
	\begin{tabular}{l >{\raggedright\arraybackslash}p{4cm} c c}
		\toprule
		\textbf{Langage} & \textbf{Caractéristique principale} & \textbf{Typage} & \textbf{Année} \\
		\midrule
		Python & Simplicité et lisibilité & Dynamique & 1991 \\
		Java & \enquote{Write once, run anywhere} & Statique & 1995 \\
		C++ & Performance et contrôle système & Statique & 1985 \\
		\rowcolor{lightgray!50} % Exemple de couleur de ligne
		\multirow{-4}{*}{\rotatebox{90}{\textbf{Populaires}}} & & & \\
		\bottomrule
	\end{tabular}
\end{table}


%======================================================================================
\section{Boîtes d'Information et Listings de Code}
%======================================================================================

\subsection{Boîtes d'Information}
Plusieurs types de boîtes colorées sont disponibles pour mettre en évidence certaines informations.

\res{
	C'est la fin de l'expérience. Le \textbf{résultat} est positif et confirme notre hypothèse de départ.
}

\comp{
	Par \textbf{comparaison}, l'approche A est 50\% plus rapide que l'approche B, mais consomme plus de mémoire.
}

\obs{
	Une \textbf{observation} importante : le système devient instable lorsque la température dépasse \SI{100}{\celsius}.
}

\warning{
	\textbf{Attention} : ne modifiez jamais les fichiers du noyau directement, au risque de corrompre le système.
}

\subsection{Listings de Code}
L'environnement \texttt{codeboxlang} permet d'afficher du code avec une coloration syntaxique adaptée au langage.

\tsec{Exemple de code Python}
\begin{codeboxlang}{python}
	# Simple script pour saluer le monde
	def say_hello(name):
	"""
	Cette fonction affiche un message de salutation.
	"""
	print(f"Hello, {name}!")
	
	if __name__ == "__main__":
	say_hello("World")
\end{codeboxlang}

\tsec{Exemple de code Java}
\begin{codeboxlang}{java}
	// Fichier: HelloWorld.java
	public class HelloWorld {
		/**
		* Le point d'entrée du programme.
		*/
		public static void main(String[] args) {
			System.out.println("Hello, World from Java!"); 
		}
	}
\end{codeboxlang}

\tsec{Exemple de requête SQL}
\begin{codeboxlang}{sql}
	-- Sélectionner les utilisateurs actifs
	SELECT
	user_id,
	user_name,
	registration_date
	FROM
	users
	WHERE
	is_active = 1
	ORDER BY
	registration_date DESC;
\end{codeboxlang}


%======================================================================================
\section{Nomenclature et Références}
%======================================================================================

\subsection{Nomenclature}
Les termes définis dans le texte avec \verb|\nomenclature| sont rassemblés ici.
\printnomenclature

\subsection{Références et Hyperliens}
Le package \texttt{hyperref} est configuré pour les liens internes et externes.
\begin{itemize}
	\item Un lien interne vers la section sur les mathématiques : voir section \ref{sec:mathématiques-et-environnements-scientifiques}.
	\item Un lien interne vers la figure des logos : voir figure \ref{fig:logo_principal}.
	\item Un lien externe vers le site du projet LaTeX : \url{https://www.latex-project.org/}.
\end{itemize}

%======================================================================================
\section{Conclusion}
%======================================================================================

Ce document a exploré avec succès une grande partie des fonctionnalités de la classe \texttt{rapport.cls}. De la mise en page à la typographie mathématique, en passant par les éléments graphiques et les extraits de code, cette classe fournit un cadre robuste et esthétique pour la rédaction de rapports professionnels et académiques.

\paragraph*{Mise en perspective}
Il est évident que la classe \texttt{rapport.cls} est un outil puissant pour la rédaction de rapports professionnels et académiques, offrant une gamme complète de fonctionnalités pour assurer une mise en page esthétique et structurée. De plus, son intégration des éléments mathématiques permet d'obtenir des documents bien formatés dans ce domaine spécifique. Enfin, la présence d'extraits de code, tels que les listings LaTeX ou le Python, offre une avance supplémentaire en termes de clarté et de facilité à la compréhension pour un large éventail de lecteurs.

\merci

\end{document}





















































